\chapter{Domande}
\paragraph{Trattamento malattie neuromuscolari}
Le malattie neuromuscolari sono malattie con decorso progressivo, che portano alla degenerazione dei muscoli. Se questo è un danno
primario, parliamo di miopatie, se invece è conseguente alla degenerazione del SNP parliamo di neuropatie.
Queste malattie sono progressive, quindi è importante limitare la progressione dei danni, cercare di rallentarli, e preparare la persona
a quello che sarà il suo futuro. In generale, si cerca di mantenere l'autonomia per più tempo possibile, mantenendo allo stesso tempo la
funzionalità muscolare. Si cerca di evitare che avvengano retrazioni miotendinee, capsulari, e che ci siano deformità osteoarticolari
dovute, tra l'altro, anche alle retrazioni e all'immobilità. \'E importante mantenere e potenziare la funzione respiratoria, anche per
mantenere l'autonomia della persona, e cercare di evitare il sovrappeso.

Sono possibili interventi chirurgici, soprattutto per ridurre le retrazioni, prolungando la deambulazione, e per correggere le scoliosi,
se si formano.

Nella riabilitazione si consiglia un'attività muscolare moderata, che permetta di mantenere la funzionalità, evitando gli esercizi di
resistenza, che peggiorano il deterioramento dei muscoli. \'E importante, invece, mantenere la partecipazione del soggetto nella vita
sociale, famigliare e con gli amici.

Le ortesi notturne possono aiutare, se il bambino cammina, a mantenere la dorsiflessione della caviglia, e dei tutori da usare durante
la deambulazione riducono le contratture. Se i bambini non camminano, si possono usare dei tutori specifici per la postura seduta.
Esistono ausili per la statica, da usare dopo la perdita della deambulazione, e la carrozzina elettronica aiuta l'autonomia negli
spostamenti.

Nelle atrofie muscolari spinali si può utilizzare anche la fisioterapia respiratoria, e ausili respiratori invasivi o meno, a seconda
della gravità della patologia, oltre che una mobilizzazione quotidiana per evitare rigidità e retrazioni. \'E possibile, nei casi più
gravi, usare anche la PEG. 

\paragraph{Gestione delle PCI}
Innanzitutto è necessario elaborare un profilo del paziente, seguendo una valutazione sia nel campo motorio (posture, funzioni, \dots),
che nel campo non motorio (natura della lesione, approccio della famiglia, disturbi sensoriali e neuropsicologici, eventuali
complicanze, \dots). Questo profilo va elaborato anche in base all'età del paziente, perché cambiando le necessità del paziente cambiano
gli ambiti di intervento sul paziente.

\'E importante che il nostro intervento sia precoce, intensivo (almeno 4 sedute da un'ora a settimana) e continuativo, per i primi anni
di vita del paziente. La valutazione va fatta seguendo anche scale e strumenti standardizzati, tenendo conto anche della possibile
prognosi del paziente. \'E importante tenere conto di obiettivi realistici, e adattare spesso il programma terapeutico, in base alle
modificazioni intermedie del paziente.

La partecipazione al programma della famiglia, e, da quando possibile, del bambino, deve essere garantita attraversso il contratto
terapeutico, che andrà a definire metodi e obiettivi da raggiungere, e andrà rinnovato nel tempo. Il programma verrà attuato in
collaborazione con un team multidisciplinare, e con la valutazione di specialisti in ogni campo. 

I genitori vanno coinvolti nel processo terapeutico, senza sostituirli al terapista, ma facendo in modo che il bambino si trovi, nella
vita di tutti i giorni, grazie alla loro collaborazione, ad affrontare esperienze utili e coerenti con la terapia.

Infine, è fondamentale una valutazione costante del lavoro svolto, sia per confronto, sia per apprendere e migliorare l'approccio da
seguire, sia per documentare i progressi o le modificazioni del paziente.
