\chapter{Paralisi cerebrali infantili}
Le PCI sono un gruppo di disordini che includono disturbi del movimento, della postura e della funzione motoria.
Derivano da lesioni del SNC con perdita di tessuto cerebrale. Possono avvenire prima, attorno o dopo la nascita, ma comunque entro il
periodo di sviluppo cerebrale del bambino. 
Sono: \textbf{persistenti}, perché la lesione non guarisce, \textbf{non progressive}, perché nemmeno progredisce la lesione,
\textbf{permanenti}, perché non guarendo la lesione non ci sarà mai un recupero completo, ma \textbf{non immodificabili}, perché possono
beneficiare della riabilitazione, ed eventualmente del trattamento chirurgico.
Oltre ai disordini di movimento si possono associare anche disordini cognitivi, sensoriali o neuropsicologici.

Sono più frequenti nei bambini prematuri e nei bambini con basso peso alla nascita, per la possibilità aumentata di andare incontro a
riduzioni del flusso cerebrale per la mancata maturazione dei meccanismi di regolazione.
I fattori predisponenti possono essere
\begin{itemize}
\item Prenatali, come fattori genetici o infezioni materne e agenti tossici durante la gravidanza.
\item Perinatali, come la prematurità, l'ipossia o l'ischemia, la postmaturità o un parto difficoltoso.
\item Postnatali, come infezioni, traumi cranici, arresti cardiocircolatori prolungati o male epilettico.
\end{itemize}

Nel prematuro, di solito, c'è un collegamento con l'emorragia intraventricolare e la leucomalacia periventricolare, con degenerazione
della sostanza bianca, vicino ai ventricoli cerebrali, che sarebbe addetta al controllo delle aree motorie e cerebrali.
Nel neonato a termine, invece, di solito c'è un collegamento con un'asfissia generalizzata, che porta a un danno all'encefalo, o con
un'occlusione di un'arteria cerebrale, con una lesione che include tutto o parte di un emisfero, e un'emiplegia più o meno grave.

Viene detta embriopatia quando si verifica entro il terzo mese di gestazione, durante la genesi degli organi, creando di solito
malformazioni varie, o fetopatia, se si verifica dopo il terzo mese di gravidanza, durante la maturazione degli organi.

A seconda delle caratteristiche del movimento che ne risultano, le PCI si dividono in:
\begin{itemize}
\item Spastiche, se c'è un costante aumento del tono di alcuni gruppi muscolari e se risultano aumentati i riflessi di stiramento.
Possono essere sia emiplegie, sia diplegie, sia tetraplegie. Sono caratterizzate da disturbi del sistema piramidale, con ipertono
flessorio degli AASS ed estensiorio degli AAII. Ci sono delle posture caratteristiche, e il movimento è ridotto, e segue schemi abnormi,
con ipereflessia e clono. I movimenti volontari sono lenti, la motricità fine è ridotta, e il bambino si stanca facilmente.
\item Atassiche, se ci sono disturbi di coordinazione ed equilibrio. Possono essere atassie semplici o diplegie. In queste forme il
danno è soprattutto cerebellare. Si associano spesso a ritardo mentale e difficoltà del linguaggio. Possono essere dovute anche a un
danno dei sistemi di collegamento tra cervello e cervelletto. Sono caratterizzate da ipotonia, instabilità nelle posture, problemi di
equilibrio, dismetria, adiadococinesia, problemi nel cammino, e difficoltà a coordinare l'azione dei vari muscoli. La diagnosi si fa a
partire dal primo anno di vita, per ipotonia, difficoltà ad evocare i riflessi, nistagmo. Spesso è coinvolta anche la bocca, mentre sono
conservate le tappe di scomparsa dei riflessi arcaici. 
\item Forme discinetiche: dipendono da una disfunzione del sistema extrapiramidale, con lesione dei gangli della base, di solito dovuta
ad asfissia perinatale. Hanno movimenti o posture abnormi, legate a un disordine della coordinazione, della regolazione del tono o di
entrambi. Non riescono a organizzare i movimenti intenzionali, a coordinare quelli automatici, e a mantenere la postura. 
\begin{itemize}
\item Distoniche, se il tono fluttua e ci sono movimenti involontari solo durante la veglia. \'E caratterizzata da ipertonia fluttuante,
distonia e ipocinesia, con instabilità nelle posture e difficoltà di coordinazione. Il tono fluttua in base alle emozioni, le posture e
i tentativi di muoversi volontariamente, con ipertono estensorio del tronco, contrazioni prolungate e movimenti distonici. Sono
possibili anche delle ipercinesie. 
\item Coreo-atetosiche, se ci sono ampi movimenti involontari. \'E caratterizzata da ipotonia e ipercinesia, con instabilità posturale e
difficoltà di coordinazione. Nel tempo diventano ipertonici. 
\end{itemize}
\end{itemize}

A seconda della sede, si dividono in:
\begin{itemize}
\item Tetraplegie, se sono coinvolti il tronco e i quattro arti. La tetraplegia spastica è spesso associata alla prematurità, con
sofferenza anossico-ischemica, mentre nel nato a termine è dovuta spesso a fattori prenatali o perinatali, con danno anossico, ischemico
o emorragico. Già alla nascita si possono riscontrare microcefalia, convulsioni, anomalie del tono e della motilità spontanea,
iporeattività, alterazione dei riflessi e persistenza di quelli arcaici. I segni neurologici includono alterazioni del tono, del
movimento, sensoriali, e contratture e deformità muscolo-scheletriche. In particolare, verso i 5-6 anni si possono creare deformità ai
piedi, mentre verso i 12 alle ginocchia. Si può andare incontro a sublussazione delle teste femorali e a scoliosi.
\item Emiplegie, se è coinvolto un solo lato. L'emiplegia congenita è la forma più comune nei nati a termine, e di solito discende da
disordini circolatori in gravidanza. Solitamente si riscontra dal quarto mese all'anno di vita, a seconda che si esaminino i riflessi e
il tono o il comportamento spontaneo del bambino. I segni neurologici più presenti sono: riduzione della mobilità e della velocità del
movimento, anomalie nel tono e nella postura, e modificazioni del muscolo e delle ossa, oltre a deficit sensoriali, movimenti speculari
e scoliosi. 	
\item Diplegie, se sono coinvolti i quattro arti, ma soprattutto quelli inferiori. La diplegia spastica è la più frequente PCI, ed è
particolarmente frequente nei pretermine. \'E dovuta a lesioni ipossico ischemiche della sostanza bianca, provocando leucomalacia
periventricolare. I fattori di rischio possono essere associati alla madre, come complicanze del parto o alto numero di gravidanze,
possono essere genetici, malformazioni, anomalie della gravidanza, o sofferenze ipossico ischemiche in condizioni non ottimali. Per la
diagnosi precoce si valuta la motilità spontanea del neonato, e di solito si esegue entro i primi mesi di vita. I segni neurologici
classici sono: alterazioni di forza e motilità volontaria, del tono, dei sensi, dei muscoli e delle articolazioni, della vista, e
disturbi cognitivi, per assottigliamento del corpo calloso. Possono presentarsi disturbi di apprendimento, come disprassia (difficoltà
di coordinazione), disgrafia e difficoltà di lettura.
\end{itemize}
Queste forme possono essere minime, moderate, lievi o gravi.

\begin{landscape}

\begin{table}[]
\centering
\caption{Tipi di Paralisi Cerebrali Infantili}
\label{Tipi}
\resizebox{1.5\textwidth}{!}{%
\begin{tabular}{@{}ccccc@{}}
\toprule
\rowcolor[HTML]{C0C0C0} 
Nome         & Caratteristiche                                                                         & Tipo                          
& Lesione                                                                             & Problemi                                        
\\ \midrule
Spastiche    & \begin{tabular}[c]{@{}c@{}}Aumento del tono e dei\\ riflessi da stiramento\end{tabular} & \begin{tabular}[c
{@{}c@{}}Emiplegie, diplegie,\\ tetraplegie\end{tabular}                                           & Sistema piramidale                
& \begin{tabular}[c]{@{}c@{}}Ipertono flessorio agli AASS, ed estensorio agli AAII, con\\ posture caratteristiche, movimento ridotto,
clono e\\ ipereflessia. Movimenti volontari lenti, facile stancabilità.\end{tabular}                                                    
\\
\rowcolor[HTML]{C0C0C0} 
Atassiche    & \begin{tabular}[c]{@{}c@{}}Problemi di coordinazione\\ e dell'equilibrio\end{tabular}   & \begin{tabular}[c]{@{}c@{}}A
volte coinvolta\\ la bocca, ritardo\\ mentale, difficoltà\\ di linguaggio.\end{tabular} & \begin{tabular}[c]{@{}c@{}}Cerebellare o dei\\
sistemi di collegamento\end{tabular} & \begin{tabular}[c]{@{}c@{}}Ipotonia, instabilità posturale, problemi di equilibrio e del\\
cammino, dismetria, adiadococinesia, difficoltà di coordinazione.\end{tabular}                                                          
\\
Discinetiche & \begin{tabular}[c]{@{}c@{}}Problemi di tono e\\ coordinazione, e posturali\end{tabular} & \begin{tabular}[c
{@{}c@{}}Distoniche o\\ coreo-atetosiche\end{tabular}                                              & Sistema extrapiramidale            
& \begin{tabular}[c]{@{}c@{}}Distonie: ipertonia fluttuante, distonia, ipocinesia, instabilità\\ posturale. Ipertono estensorio del
tronco, movimenti distonici.\\ Coreo-atetosi: ampi movimenti involontari, con ipotonia,\\ ipercinesia, instabilità posturale e
difficoltà di coordinazione.\\ Possibile un'evoluzione in ipertonia.\end{tabular} \\ \bottomrule
\end{tabular}%
}
\end{table}

\begin{table}[]
\centering
\caption{Localizzazione delle Paralisi Cerebrali Infantili}
\label{Local}
\resizebox{1.5\textwidth}{!}{%
\begin{tabular}{@{}ccccc@{}}
\toprule
\rowcolor[HTML]{C0C0C0} 
Nome        & Localizzazione                                                                                   & Cause                  
& Diagnosi                                                                                                                           &
Segni neurologici                                                                                                                      
\\ \midrule
Tetraplegie & \begin{tabular}[c]{@{}c@{}}Coinvolgimento di tutti\\ gli arti e del tronco\end{tabular}          & \begin{tabular}[c
{@{}c@{}}Nel prematuro: sofferenza\\ anossico ischemica\\ Nel nato a termine: danno anossico,\\ ischemico o emorragico.\end{tabular} &
Alla nascita riscontrabili microcefalia, distonie, convulsioni, \dots                                                                &
\begin{tabular}[c]{@{}c@{}}Alterazioni del tono, del\\ movimento, sensoriali,\\ deformità muscolo scheletriche.\end{tabular}            
\\
\rowcolor[HTML]{C0C0C0} 
Emiplegie   & \begin{tabular}[c]{@{}c@{}}Coinvolgimento di un\\ solo emisoma\end{tabular}                      & Congenita, o con
disordini circolatori in gravidanza.                                                                                                &
\begin{tabular}[c]{@{}c@{}}Dal quarto mese all'anno di età,\\ controllando riflessi, tono, e\\ comportamento spontaneo.\end{tabular} &
\begin{tabular}[c]{@{}c@{}}Riduzione della mobilità e\\ della velocità di movimento,\\ anomalie di tono e postura,\\ deformità,
scoliosi,\\ deficit sensoriali.\end{tabular} \\
Diplegie    & \begin{tabular}[c]{@{}c@{}}Coinvolti i quattro arti,\\ soprattutto quelli inferiori\end{tabular} & \begin{tabular}[c
{@{}c@{}}Lesioni ipossico-ischemiche della sostanza bianca,\\ con leucomalacia\\ periventricolare.\end{tabular}                    &
\begin{tabular}[c]{@{}c@{}}Entro i primi mesi, valutando la\\ motilità spontanea.\end{tabular}                                     &
\begin{tabular}[c]{@{}c@{}}Alterazioni di: forza, motilità,\\ tono, sensi, muscoli,\\ articolazioni, vista.\\ Disturbi
cognitivi.\end{tabular}                               \\ \bottomrule
\end{tabular}%
}
\end{table}

\end{landscape}

\section{Disturbi associati}
\paragraph{Disturbo intellettivo}
Non sempre è correlato alle PCI, ma compare in due terzi dei casi, con lesioni estese, mentre non compare spesso in quelle localizzate e
lateralizzate. \'E presente di solito nella tetraplegia, più che nella emiplegia, e più nelle forme spastiche che in quelle
discinetiche. Spesso è difficile da quantificare per la contemporanea presenza del disturbo motorio, e spesso è la stessa patologia
motoria a impedire un corretto sviluppo cognitivo, per la povertà delle esperienze.

\paragraph{Disturbo percettivo}
La percezione è fondamentale per costruire un movimento corretto, così come il movimento è fondamentale per raccogliere le informazioni
in modo appropriato. Questi due aspetti, che si influenzano a vicenda, vanno a influenzare molto il recupero. Può essere dovuto a un
problema di attenzione, ossia le informazioni vengono raccolte, ma vengono poi inserite in modo automatico nel controllo motorio e
posturale, o può essere dovuto a problemi di tolleranza: soggetti con una soglia di tolleranza bassa hanno problemi a muoversi se gli
stimoli superano la loro soglia di tolleranza. In questo caso, il programma motorio sarebbe realizzabile, ma viene compromesso
dall'incapacità di tollerare le varie informazioni, creando un dispiacere nel movimento, e creando quindi una paralisi intenzionale. Le
paralisi possono quindi essere motorie, percettive, o intenzionali. Influiscono sia sullo sviluppo iniziale del movimento, sia
sull'apprendimento.

\paragraph{Disturbo sensoriale}
Sono anomalie nella vista (presenti in metà dei soggetti) o nell'udito (presenti raramente). Sono spesso associati alla leucomalacia
periventricolare.

\paragraph{Disturbo prassico}
\'E caratterizzato dall'incapacità di compiere movimenti volontari, o una sequenza di movimenti per uno scopo. C'è anche difficoltà a
tradurre le proprie intenzioni in una serie di movimenti. 

\paragraph{Disturbo della parola e del linguaggio}
Spesso ci sono difficoltà ad articolare le parole: anartria (problemi a programmare l'articolazione, per una lesione sinistra),
disartria (disturbo motorio del linguaggio) e altri. Spesso il linguaggio è poco comprensibile, ma spesso è anche per colpa del deficit
cognitivo. 

\paragraph{Disturbo comportamentale}
Consistono in emotività eccessiva, ADHD, e disturbo ossessivo-compulsivo.

\paragraph{Accrescimento}
Risulta ridotto, per via del poco cibo ingerito, del vomito, del reflusso e della paralisi pseudobulbare. Alcune volte c'è il rischio di
fratture spontanee, per osteopenia e osteoporosi.

\paragraph{Epilessia}
\'E presente in un terzo o due terzi dei soggetti, si manifesta presto, nei primi due anni di vita, e spesso è resistente ai farmaci.
\'E più frequente nelle tetraplegie spastiche e nelle emiplegie postnatali, e meno nelle diplegie spastiche simmetriche e nelle forme
atetosiche. Ha prognosi migliore con intelligenza normale e singoli episodi convulsivi.

\paragraph{Attività}
A tre mesi il bambino con PCI non solleva il capo, non spinge sulle braccia, estende il capo per riflesso tonico labirintico e non porta
le mani sulla linea mediana per riflesso tonico asimmetrico.

A sei mesi non solleva il capo, ha il dorso curvo, le braccia rigide e le mani a pugno, non controlla il capo da seduto, ha le braccia
flesse e le gambe rigide. 

A nove mesi usa poco le mani, ha il dorso curvo, le gambe rigide e i piedi equini, e non regge il carico in piedi.

A un anno usa solo un emisoma, non gattona, le gambe sono rigide e ha difficoltà a tenersi in verticale. 
\begin{itemize}
\item Il riflesso tonico asimmetrico impedisce il rotolone, di portare le mani alla bocca e sulla linea mediana.
\item Il riflesso tonico simmetrico impedisce una corretta andatura a gatto e il carico su un avambraccio per liberare l'altro.
\item Il riflesso tonico labirintico impedisce l'estensione del capo e il carico sugli avambracci, e il mantenimento della posizione
prona. 
\item Il grasp palmare impedisce il rilascio, la maturazione della presa e l'esplorazione di oggetti.
\item Il griff del piede impedisce il mantenimento dell'equilibrio.
\item Il riflesso gamba su bamba impedisce l'andatura a gatto e il cammino.
\item Il riflesso del succhio impedisce masticazione e articolazione.
\end{itemize}

\section{Caratteristiche delle PCI}
\paragraph{Tetraparesi spastica}
La tetraparesi spastica è caratterizzata da ipertono ai quattro arti, ipotono nella regione nucale, disartria e disfagia, ritardo
cognitivo, e difficoltà nel controllo del capo e nel carico sui gomiti (da prono), difficoltà a portare le mani sulla linea mediana,
schemi patologici agli arti, che condizionano i movimenti del soggetto, e dalla persistenza dei riflessi arcaici.

Il movimento si caratterizza per l'impossibilità del rotolone, per l'assenza dello svincolo dei cingoli e con una possibile caduta nel
passaggio da prono a supino. Lo striscio è possibile solo nei casi più lievi, con il solo aiuto degli avambracci, e la postura seduta è
difficile per un controllo incompleto del capo e del tronco, che si trova cifotico ed asimmetrico e per gli arti inferiori flessi.
La stazione quadrupedica è possibile solo nei casi più lievi, per la flessione del capo, degli arti inferiori e dei gomiti, e quando
avviene, l'andatura è difficoltosa e non fluida.
La stazione eretta e il cammino sono possibili solo con sostegno, e difficilmente sono funzionali.

I problemi che caratterizzano la patologia sono la maggiore compromissione di un lato, che porta ad asimmetria nelle posture, il fatto
che si creano retrazioni e deformità, il bambino non ha nemmeno la motivazione a muoversi, e ci sono problemi con la parola, la
deglutizione e la respirazione. 

\paragraph{Diplegia spastica}
La diplegia spastica è caratterizzata da ipertono degli arti inferiori, impaccio dei superiori, disturbi della percezione e delle
prassie. Il bambino sviluppa il controllo del capo e il suo orientamento sulla linea mediana, usa le mani sulla linea mediana, ma ha gli
arti inferiori ipomobili e in estensione, e cerca di supplire muovendosi tramite gli arti superiori. Gli arti inferiori non sono in
grado di effettuare movimenti indipendenti, e passano dallo schema estensorio a quello flessorio. 

Il movimento è caratterizzato da difficoltà nel rotolone per scarso svincolo dei cingoli e aumento del tono estensorio. Lo striscio è
reso possibile dall'uso degli arti superiori, con gli inferiori estesi. Da seduto non ha un buon controllo del tronco, gli arti
inferiori restano estesi, con l'anca poco flessa, e si compensa per cercare equilibrio con una pronunciata cifosi del dorso. Per
mantenere orizzontale lo sguardo, aumenta la lordosi cervicale, e a causa della mancanza di reazioni di sostegno, queste vengono
supplite dagli arti superiori. 
L'andatura a gattoni è possibile, anche se non con movimento indipendente degli arti inferiori, con aumento della flessione di anche e
ginocchia, e piccoli spostamenti delle gambe.
La stazione eretta è possibile con un appoggio stabile per gli arti superiori, con appoggio in punta e arti inferiori estesi. La postura
spesso è asimmetrica, con un ginocchio iperesteso e uno flesso, entrambi addotti e intraruotati. 
Il cammino è possible con sostegno fisso, per difficoltà nello spostamento del carico, base d'appoggio ristretta, incapacità di fermarsi
e scarso equilibrio.

I problemi sono le retrazioni, le deformità, il ritardo nell'andare in carico, con un cammino che resta poco funzionale, aumento del 
peso e poca motivazione a muoversi. 

\paragraph{Emiparesi spastica}
L'emiparesi è caratterizzata da ipertono ad un emisoma, con disturbi della sensibilità, asimmetria del carico ed eventualmente epilessia
e disturbi del movimento.

La posizione prona non è gradita, perché i gomiti non caricano in modo simmetrico. Il rotolone è consentito solo verso il lato paretico,
ma è difficile liberare quell'arto. Lo strisciamento è fatto con un arto sotto il tronco, e la gamba estesa e intraruotata. Da seduto il
carico è tutto sul lato sano, la gamba paretica resta flessa e sono assenti le reazioni di appoggio dal lato paretico. Il passaggio da
supino a seduto è possibile solo dal lato sano, con aumento del tono. Lo spostamento in quadrupedica è possibile solo nelle forme lievi,
mentre si trascina da seduto usando il lato sano. Per raggiungere la stazione eretta si mette in ginocchio (in modo asimmetrico), poi si
appoggia con le braccia, inizia il movimento col lato paretico e lo conclude con quello sano.
La stazione eretta è mantenuta con il carico sul lato sano, il tronco inclinato dall'altro lato, il bacino elevato, i piedi in punta,
l'anca e il ginocchio flessi.
Il cammino viene portato avanti con una rotazione del bacino, un'extrarotazione dell'arto inferiore, l'iperestenzione del ginocchio, e
il tono aumentato nel braccio.

I problemi principali sono che tutte le attività vengono fatte col lato sano, l'altro resta ipotrofico, si creano delle retrazioni,
dalla parte paretica ci sono deficit sensitivi, e di solito non svolgono volentieri i compiti assegnati.

\paragraph{Tetraparesi distonica}
Questa sindrome è caratterizzata da tono fluttuante ai quattro arti, e ipotono al dorso. La postura è instabile, ci sono dei movimenti
involontari, problemi percettivi, disfagia, disartria, ipotonia, mancato controllo del capo, del tronco, riflessi arcaici mantenuti,
arti inferiori flessi, ma con buone cpacità intellettive e di relazione.

La posizione supina ha delle improvvise accentuazioni del tono estensorio, il capo resta ruotato, con riflesso tonico asimmetrico e
grasp.
Da prono non controlla il capo e non si appoggia ai gomiti.
\'E in grado di rotolare da un solo lato, iniziando dalle gambe.
Resta seduto con flessione o estensione del tronco e del capo, il tronco tutto da un lato, e gli arti superiori poco coordinati.
Raggiunge la posizione a gatto con il riflesso tonico simmetrico, e poi procede a canguro.
La stazione eretta viene raggiunta tardi, anche con un sostegno non fisso, con carico asimmetrico e molti compensi.
Il cammino viene raggiunto tardi, con un'andatura a scatti, evitando di cedere con le gambe, spesso iperestendendo.
Usando il grasp e il riflesso tonico asimmetrico è in grado di avere una manualità di base, con strategie specifiche per permettere il
rilascio, usando una sola mano alla volta.

I principali problemi sono una postura non stabile, i compensi, che non sempre sono stabili, e possono portare a contratture, una bassa
efficienza, e la necessità di ausili per apprendimento e comunicazione.

\paragraph{Forme atassiche}
Sono caratterizzate da ipotonia, alterazioni dell'equilibrio, della marcia, del gesto e della parola, ma anche del movimento oculare.

La postura supina è mantenuta senza schemi patologici, ma condizionata dalla gravità, la postura prona ha un ritardo del carico sugli
avambracci.
Il rotolone viene raggiunto in ritardo, e lo striscio non ha schemi patologici, tanto che viene usato molto a lungo.
La postura seduta viene acquisita in ritardo, è poco stabile, con poco equilibrio.
L'andatura a gatto è rallentata, poco coordinata, e la stazione eretta ha grossi problemi di equilibrio.
Il cammino viene raggiunto in ritardo e con un appoggio, con problemi di equilibrio e base allargata.
La manualità è grossolana, condizionata dalla postura, e dal peso dell'oggetto.

I problemi principali sono un ritardo nell'acquisizione dei vari passaggi posturali, la poca efficacia delle reazioni di equilibrio,
l'instabilità delle posture e dell'equilibrio, la scarsa autonomia nel cammino e nelle ADL.

\paragraph{Schemi patologici}
Gli arti superiori vengono valutati a seconda della posizione delle spalle.
Possono essere estesi, se le spalle sono indietro, con braccia abdotte e extraruotate, gomiti flessi, avambracci pronati e polsi flessi,
o flessi, con spalle anteposte, braccia addotte e intraruotate, gomiti estesi, avambracci pronati e polso flesso.

Gli arti inferiori possono avere lo schema flessorio, se c'è flessione, abduzione e rotazione esterna delle anche, flessione delle
ginocchia, flessione plantare e piedi supinati, o estensorio, se le anche sono estese, addotte e ruotate internamente, le ginocchia
estese e i piedi in punta.

Spesso ci sono contratture degli estensori del capo, flessori del gomito e del polso, estensori lombari, flessori di anca, ginocchio e
piede, ginocchio e piede valgo.

La schiena spesso è in cifosi, con lordosi lombare, lussazione dell'anca e piede piatto. L'anca viene lussata più facilmente se il collo
del femore è valgo, il tetto dell'acetabolo è poco pronunciato, gli adduttori sono ipertonici e gli abduttori ipotonici, e il carico
viene raggiunto in ritardo.
