\chapter{Il nato pretermine}
Il bambino viene detto nato pretermine se nasce prima della 37sima settimana di gravidanza. Spesso, ma non sempre,
questa condizione si associa con il basso peso alla nascita, inferiore ai 2.5 kg. Se le due condizioni avvengono
contemporaneamente, aumentano i rischi per la salute del bambino, e aumenta il rischio di mortalità, per difficoltà
nelle funzioni vitali, problemi a regolare la temperatura corporea, ad alimentarsi, a respirare e a svilupparsi.

L'età gestazionale viene calcolata in settimane complete dal primo giorno dell'ultima mestruazione. Se un bambino
nasce prima delle 37 settimane è pretermine, se nasce tra 37 e 42 è a termine, se nasce dopo le 42 è post-termine.
I possibili fattori di rischio sono:
\begin{itemize}
\item Fetali:
\begin{itemize}
\item Sofferenza fetale
\item Gravidanza multipla
\end{itemize}
\item Placentari:
\begin{itemize}
\item Placenta previa
\item Distacco di placenta
\end{itemize}
\item Uterine: Anomalie congenite
\item Materne:
\begin{itemize}
\item Pre-eclampsia (gestosi)
\item Patologie croniche
\item Infezioni
\item Abuso di alcool, fumo e droghe
\end{itemize}
\item Altre: Rottura prematura delle membrane
\end{itemize}

Il peso si divide in basso (<2500g), molto basso (<1500g), molto molto basso (<1000g), estremamente basso (<750g).
I fattori di rischio sono:
\begin{itemize}
\item Fetali:
\begin{itemize}
\item Alterazioni cromosomiche
\item Anomalie congenite
\item Gestazione multipla
\item Infezioni intrauterine
\end{itemize}
\item Placentari: Anomalie della placenta, come difficoltà a nutrire adeguatamente il feto
\item Materne:
\begin{itemize}
\item Malattie croniche o insorte in gravidanza, come cardiopatie, nefropatie, anemie, pre-eclampsia, malattie respiratorie.
\item Abuso di alcool, fumo o droghe
\end{itemize}
\end{itemize}

Il bambino nato pretermine è più esposto del bambino nato a termine a sviluppare delle patologie che possono essere:
\begin{itemize}
\item Maggiori:
\begin{itemize}
\item Paralisi cerebrale infantile grave
\item Insufficienza mentale grave
\item Deficit visivo grave, come cecità, ipovisione grave, o atrofia del nervo ottico
\item Deficit uditivo grave, come ipoacusia o anacusia
\item Eventualmente associate tra loro, e a idrocefalo o epilessia
\end{itemize}
\item Minori
\begin{itemize}
\item Ritardi mentali lievi
\item Paralisi cerebrali infantili lievi, come la diplegia
\item Disfunzione cerebrale minima
\end{itemize}
\end{itemize}

Le diverse patologie possono essre dovute a tre cause diverse: patologie del sistema nervoso centrale (ipossie,
ischemie, emorraggie, malformazioni e deficit nutrizionali), esposizione precoce a un ambiente psichico, fisico
e sociale a cui il bambino non era preparato, e non ottimale per lo sviluppo del SNC, o interruzione e disturbo
dello sviluppo della relazione del bambino con la madre e la famiglia.

Il nato pretermine ha uno sviluppo particolare, a seconda del tipo di sofferenza che ha portato alla nascita
prematura, e a seconda del momento della nascita. Solitamente manifesta una patologia ipossico-ischemica, che porta
a leucomalacia periventricolare e a paralisi diplegica, mentre il nato a termine è più soggetto a un danno dei nuclei
della base, con paralisi cerebrale infantile distonica o atetosica.
Questo sviluppo è dovuto alla mancanza di un utero materno che filtri i rapporti con l'esterno, proteggendo il bambino,
alla mancanza dei corpi dei genitori, con un'accoglienza e un rapporto particolari, e alla mancanza di un gruppo sociale.
In terapia neonatale, infatti, il bambino pretermine si trova sottoposto a una separazione dalla madre e a molte
esperienze sensoriali e dolorose diverse dal fisiologico, che rischiano di sovraccaricarlo.

Nelle prime ore dalla nascita, i problemi più gravi sono quelli della stabilizzazione delle funzioni vitali omeostatiche,
per la respirazione, la circolazione, la termoregolazione, \dots

Nel periodo successivo, invece, è fondamentale promuovere un sonno adeguato e avere cura delle posture del bambino.

Esiste una \textit{teoria sinattiva}, per cui il bambino pretermine è un bambino competente per la vita intrauterina, non
incompetente per la vita extrauterina. Secondo questa teoria, il funzionamento del neonato è regolato da 5 sottosistemi:
\textbf{neurovegetativo} (per le funzioni di base), \textbf{motorio} (movimenti e posture), \textbf{stati comportamentali}
(nella loro stabilità, appropriatezza e nei passaggi tra stati), \textbf{attenzione} (per la qualità della vigilanza e la
capacità di stare attento), e \textbf{autoregolazione} (capacità di regolare la stabilità dei sottosistemi e i loro rapporti).
I sottosistemi maturano in sequenza, e il funzionamento di uno dipende dalla funzionalità degli altri, e dalle facilitazioni
fornite al bambino. Inizialmente maturano il neurovegetativo e il motorio, gli altri seguono. In particolare il sistema
motorio, se facilitato opportunamente, stabilizza le funzioni vegetative e cardiorespiratorie, in particolar modo il flusso
ematico. La stabilizzazione degli stati comportamentali facilita l'organizzazione delle strutture neurofisiologiche.
L'autoregolazione del sistema neurovegetativo è mostrata da respirazione regolare, colorito stabile e roseo, ruttini tranquilli
e assenza di tremori e cloni. Lo stress, al contrario, è mostrato da respiro accelerato, rallentato o con pause, colorito
instabile, pallido, rosso, violaceo o a chiazze, rigurgito, vomito, emissione di urine e feci al momento del ruttino, e presenza
di tremori e cloni.

\paragraph{La valutazione dello sviluppo}
Si devono valutare le funzioni dello sviluppo, lo stato attuale, il potenziale evolutivo e la modificabilità della situazione.
Fino ai due anni la valutazione va fatta tenendo conto dell'età corretta, quindi togliendo dall'età anagrafica le settimane
che mancavano per arrivare alle 40 di gestazione.
Per la valutazione, un aspetto sono i general movements: sono movimenti che coinvolgono tutto il corpo, variano, esaurendosi
gradualmente, e sono fluidi, se il neonato è sano. Sono presenti fin dalla dodicesima settimana di gravidanza. Fino a due mesi
sono nella fase di \textit{writhing}, per poi passare al \textit{fidgety} dai due ai cinque mesi. Hanno una buona validità
diagnostica e prognostica, e vanno a integrare l'esame nerologico tradizionale.

La postura corporea può essere usata con scopo di prevenzione, abilitazione delle capacità residue, per permettere una
riorganizzazione delle funzioni a seguito di eventi più o meno stressanti, o con scopo riabilitativo, per le lesioni cerebrali
del bambino.
La cura posturale, quindi, ha come obiettivi \textbf{riproporre un confine} per orientarsi nello spazio e sviluppare lo schema
corporeo, \textbf{stabilizzare il sottosistema motorio} per aiutare gli altri sottosistemi, \textbf{promuovere la flessione
fisiologica}, \textbf{migliorare la coordinazione e il controllo del movimento}, \textbf{promuovere esperienze sensoriali},
\textbf{facilitare la simmetria}, \textbf{evitare gli schemi patologici}, \textbf{facilitare le competenze antigravitarie}, e
\textbf{promuovere le relazioni}.

Il contenimento deve essere adattabile a tutte le situazioni, con un sistema di postura specifico. Questi adattamenti vanno
forniti con tempi compatibili al bambino e alla famiglia, per facilitare l'organizzazione sulla linea mediana, il controllo del
capo e le competenze visive.
La postura va curata durante l'alimentazione, favorendo il contatto con la madre e peomuovendo un'esperienza piacevole. Bisogna
fare attenzione al contesto in cui si usa il sistema di postura e alle attività che vi si svolgono.
