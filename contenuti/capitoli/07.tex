\chapter{Torcicollo miogeno congenito}
Il torcicollo miogeno congenito è una patologia primaria idiopatica dello sternocleidomastoideo, che va incontro a fibrosi o altre
alterazioni e si accorcia, unilateralmente. Si rileva alla nascita o entro pochi mesi.

Le possibili cause sono un trauma alla nascita con stiramento del muscolo, accorciato congenitamente, ischemia del muscolo per
ostruzione venosa, cattive posizioni nell'utero, anche per cause genetiche, neurogene o infettive.

I segni clinici sono inclinazione dal lato affetto e rotazione dal lato opposto della testa, accorciamento del muscolo, limitazione dei
movimenti, e asimmetrie craniofacciali.
Può essere presente o meno la tumefazione della zona. Se c'è viene rilevata verso i 12 giorni d'età all'interno della massa muscolare.
Può aumentare fino alle 6 settimane, per poi ridursi e spesso sparire. Dopo i sei mesi le fibre si possono normalizzare o può comparire
fibrosi, con fibre disomogenee e ipoecogene.
Se la fibrosi non è presente, la diagnosi è più tardiva, notando la postura in torcicollo, la plagiocefalia, e la contrattura del
muscolo.

La diagnosi differenziale si fa sulla base di ecografie in serie, per differenziarlo dal torcicollo posturale acquisito.
Il TMC porta a esiti a distanza, come postura caratteristica, limitazione dei movimenti, e anomalia della crescita ossea del cranio per
gli scarsi movimenti.
Gli esiti sono diversi a seconda della forma iniziale, della severità iniziale, e della terapia.

La terapia mira a mantenere l'elasticità del muscolo e ridurre la possibilità di fibrosi con esiti permanenti. Il trattamento deve
essere preventivo, perchè è più facile ed efficace in giovane età. I familiari devono essere coinvolti, con un trattamento domiciliare, 
per avere un'intensità maggiore e un maggiore rispetto dei ritmi del bambino.


Il nostro lavoro è aiutare i genitori a capire le attività, aiutarli ad eseguirle nel modo corretto, verificando il programma
domiciliare e cercando soluzioni ai vari problemi che si presentano, e a volte integra il programma domiciliare con sedute in
ambulatorio.
Il programma si basa su cura posturale durante il sonno (passiva), durante la veglia (attiva), e stretching e allungamento muscolare
(passiva e attiva). Fino a 6-7 mesi il trattamento è intensivo, per poi ridursi nel tempo.

\chapter{Plagiocefalie}
Le \textbf{craniostenosi} sono determinate dalla chiusura precoce di una o più suture craniche, portando a una alterazione della forma
della testa. Le suture permettono l'accrescimento del cervello, si fondono verso il terzo anno di vita e successivamente si ossificano.
Uniscono le ossa frontali, parietali e occipitali.

Le craniostenosi si dividono in:
\begin{itemize}																							%Apro 1
\item Vere																								%punto 1.1
	\begin{itemize}																						%		apro 2
	\item Semplici																						%		punto 2.1
		\begin{itemize}																					%			apro 3
		\item Scafocefalia																				%			punto 3.1
		\item Plagiocefalia posteriore																	%			punto 3.2
		\item Pachicefalia (plagiocefalia posteriore bilaterale)										%			punto 3.3
		\end{itemize}																					%			chiudo 3
	\item Complesse																						%		punto 2.2
		\begin{itemize}																					%			apro 3
		\item Sindromiche																				%			punto 3.1
			\begin{itemize}																				%				apro 4
			\item Sindrome di Crouzon																	%				punto 4.1
			\item Sindrome di Apert																	%				punto 4.2
			\item Sindrome di Pfeiffer																	%				punto 4.3
			\item Sindrome di Saethre-Chotzen														%				punto 4.4
			\item \dots																					%				punto 4.5
			\end{itemize}																				%				chiudo 4
		\item Non sindromiche																			%			punto 3.2
			\begin{itemize}																				%				apro 4
			\item Plagiocefalia anteriore																%				punto 4.1
			\item Trigonocefalia																			%				punto 4.2
			\item Brachicefalia																			%				punto 4.3
			\item Oxicefalia																				%				punto 4.4
			\end{itemize}																				%				chiudo 4
		\end{itemize}																					%			chiudo 3		
	\end{itemize}																						%		chiudo 2
\item Pseudocraniostenosi, come ad esempio microcefalia, poroencefalia, 						%
derivazioni liquorali, ischemiche, posturali. 															%punto 1.2
	\begin{itemize}																						%			apro 3
	\item Malattie da accumulo (Sindromi di Hurler e di Morquio)									%			punto 3.1
	\item Malattie metaboliche (osteomalacia o ipotiroidismo)										%			punto 3.2
	\item Malattie ematologiche (policitemia vera o talassemia)									%			punto 3.3
	\item Farmacologiche (acido retinoico o difenilidantoina)										%			punto 3.4
	\item Iatrogene (shunt liquorale, \dots)															%			punto 3.5
	\item Posizionali																						%			punto 3.6
	\item Microcefalia vera																				%			punto 3.7
	\end{itemize}																						%			chiudo 3
\end{itemize}																							%		chiudo 2
Se invece le suture non si chiudono, ma si riscontra comunque un difetto di forma, siamo si fronte alle \textbf{deformazioni posturali}.

I fattori di rischio sono: prematurità, tipo e lunghezza del parto, posizioni strane alla nascita, anomalie dell'orecchio, torcicollo,
anomalie cervicali, ematoma cefalico, anomalie dell'utero, gemellarità.

La plagiocefalia posizionale si divide in:
\begin{itemize}
\item Lieve, con un appiattimento unilaterale localizzato.
\item Moderata, con uno spostamento dell'orecchio dal lato condiderato, una protrusione della fronte dal lato considerato e un
appiattimento dall'altro.
\item Severa, con gravi asimmetrie della parte posteriore, e una protrusione dell'occipite controlateralmente alla plagiocefalia.
\end{itemize}

Per l'esame, si cercano visivamente asimmetrie del volto, e dall'alto si cercano appiattimenti o protrusioni frontali e dell'occipite, e
difetti nelle proporzioni.
Manualmente, si palpano le varie suture, cercando masse fibrose associate al TMC, e prima e dopo il trattamento bisogna misurare il
cranio.

Con il posizionamento prono in culla erano frequenti le plagiocefalie anteriori, successivamente per ridurre la morte in culla si è
passati al posizionamento supino, aumentando di moltissimo le plagiocefalie posteriori, perché non venivano modificate abbastanza spesso
le posizioni del bambino, girando il capo da sinistra a destra, importante perché il peso del corpo può essere sufficiente per deformare
le ossa molli del neonato.

Per trattare la plagiocefalia posizionale si esegue un trattamento posturale e ortesico nel primo anno di vita, poi il miglioramento è
spontaneo, ed eventuali residui possono essere mascherati dai capelli. Nei casi più gravi si può eseguire un rimodellamento chirurgico.
Per prevenirla è importante alternare le posizioni della testa nel decubito supino, mentre quando il bambino impara a girarsi
autonomamente bisogna riposizionarlo supino per ridurre il rischio di morte in culla. Si può usare la posizione sul semifianco, con gli
appositi cuscini \textit{nanna sicura}.

Il lettino va orientato in modo da variare il lato del bambino esposto agli stimoli, luminosi e sonori. Al risveglio del bambino lo si
posiziona prono, per svolgere esercizio contro il torcicollo e per rinforzare la muscolatura del capo e del collo. Durante il gioco, il
bambino può restare sul fianco.

Anche in braccio è importante alternare il lato su cui sta il bambino, e favorire il movimento della testa nei momenti di interazione. I
giocattoli vanno spostati negli ambienti in cui passa più tempo, per favorire lo spostamento, e allo stesso modo va variato il lato da
cui si imbocca il bambino.

\chapter{Il cammino sulle punte}
Il termine \textit{cammino sulle punte} viene usato per descrivere un pattern regolare di cammino sulle punte, senza una causa nota.
Non ci deve essere una sofferenza neurologica, spesso è dovuto o causa una retrazione del tendine d'achille, ed è ereditario, infatti
spesso risulta presente anche nelle due generazioni precedenti a quella considerata.

La visita neurologica deve escludere una lieve diplegia spastica, e varie malattie neuromuscolari. La diagnosi può essere fatta quando
il bambino inizia a camminare ad un'età normale, presentando il pattern, ma con tono, forza muscolare, riflessi e sensibilità normali.
Le qualità della deambulazione, come equilibrio e coordinazione, sono buone.

Il trattamento varia in base all'età e alla presenza di limitazioni nella flessione dorsale del piede.
Se non c'è retrazione del tendine si usano esercizi di stretching e posizionamento notturno in docce.
Se c'è una lieve tensione del tendine o una contrattura del gastrocnemio si usano delle ortesi gamba piede per aiutare il contatto del
tallone col suolo, e eventualmente dei gessi caviglia piede per allungare il tendine.
Se i bambini sono più grandi e ormai il tendine è retratto si va incontro ad un intervento per l'allungamento. 

\chapter{Disfunzioni neurologiche minori}
Si sta cercando di definire la relazione tra il danno cerebrale precoce, la disfunzione neurologica, e i disordini di comportamento e
apprendimento.
Le sindromi da danno cerebrale minimo mostrano danni sia neurologici che comportamentali, quindi danni dello sviluppo motorio,
cognitivo, comportamentali e dell'apprendimento. Alcune di queste sindromi sono: la sindrome del bambino goffo, l'instabilità
psicomotoria, i disordini percettivo-motori, le disabilità specifiche della lettura e il ritardo nel linguaggio e nell'apprendimento.
Non si trova una correlazione specifica tra l'estensione del danno cerebrale e la gravità della disfunzione che ne risulta.

La coordinazione motoria è la capacità che ci permette di eseguire i movimenti in modo efficace e aderente all'immagine motoria
elaborata nel cervello.
Il disturbo di sviluppo della coordinazione motoria si manifesta quando le prestazioni sono sotto al livello atteso per età e sviluppo
intellettivo. Chi ha questo disturbo ha anche difficoltà a imparare le strategie per risolvere i problemi, quindi non hanno capacità
motorie automatiche, e devono costantemente sforzarsi per portare a termine le attività motorie.
Le persone normali, nei gesti abituali non hanno bisogno di usare attenzione per lo svolgimento, mentre per i gesti nuovi devono
selezionare la sequenza dei gesti, controllarne lo svolgimento ed eventualmente modificarlo.

Chi ha disprassia ha difficoltà a rappresentare, programmare ed eseguire gli atti motori per le azioni finalizzate, con problemi sia per
la coordinazione motoria, sia per le varie ADL. I disturbi della \textbf{coordinazione} si realizzano in un \textit{non sapere come
fare} le cose, mentre i disturbi del \textbf{controllo} ed esecuzione si realizzano in un \textit{non sapere cosa fare}.

Clinicamente, la disprassia può essere primaria, se non associata ad altre patologie e senza segni neurologici evidenti, o secondaria,
se associata ad altre patologie, come PCI, sindrome di Williams, di Down, \dots

Esistono vari tipi di disprassia, e possono comparire singolarmente o associati, con uno dei tipi preminente sugli altri. I vari tipi
sono: generalizzata, verbale, orale, dell'abbigliamento, degli AASS, della scrittura, dello sguardo, della marcia, del disegno,
costruttiva.

I sintomi sono eterogenei, ma il disturbo motorio coinvolge le abilità grosso-motorie, le fini, la coordinazione oculo-manuale, il
controllo della postura. Questi disturbi interferiscono con l'apprendimento scolastico e le ADL, e mettono a rischio la carriera
scolastica, la socializzazione, le relazioni con gli altri e l'emotività del soggetto.

La diagnosi si basa su una valutazione funzionale multidisciplinare, e deve comprendere:
\begin{itemize}
\item Esame neurologico per cercare segni di disfunzione neurologica minore correlati al disturbo motorio
\item Valutazione neuropsicologica per analizzare le abilità visuali percettive e motorie, linguistiche e cognitive
\item Valutazione delle abilità motorie
\item Valutazione delle abilità manipolatorie fini
\item Valutazione delle prassie gestuali
\item Valutazione della percezione cinestesica
\item Valutazione delle abilità costruttive
\item Valutazione del linguaggio
\end{itemize}

A seconda degli specifici disturbi coinvolti, la terapia va adattata, eventualmente con controlli regolari per cogliere disfunzioni
nell'età scolastica. Le terapie psicomotorie e quelle percettive motorie danno risultati positivi, ma che non bastano, e le indicazioni
terapeutiche permettono l'acquisizione di abilità specifiche, connesse alla vita del bambino, per aiutarlo a scomporre le sequenze e
semplificarle, sostenendolo nell'apprendimento, nella memorizzazione e nell'automatizzazione.
Le attività motorie non specifiche e la pratica sportiva non solo non sono efficaci, ma provocano ansia e frustrazione.
