\chapter{Metodi e tecniche di trattamento}
Negli anni 50 il trattamento delle PCI era analogo a quello della poliomielite, e si concentrava sull'aspetto osteo-articolare e
muscolare, per evitare deformità e retrazioni. Il trattamento non era precoce, iniziava a tre anni.
In questo periodo, Phelps ha creato una dottrina basata su tecniche di rilassamento, rinforzo muscolare, mobilizzazione passiva e
ortesi. I movimenti venivano condizionati tramite l'ascolto di una canzone specifica per ogni tipo di esercizio.

Negli anni 60 si è iniziato a studiare le PCI in modo specifico, chiarendo che a causa della lesione cerebrale veniva impedito il
normale sviluppo e apprendimento motorio e che portavano invece alla persistenza di schemi motori abnormi e primitivi.
Secondo gli studiosi di questo periodo lo sviluppo della postura e della stabilità è dovuto alla maturazione di strutture corticali che
controllano i riflessi posturali e le reazioni di raddrizzamento, di origine sottocorticale e spinale.

I metodi sviluppati in questo periodo si basano su una concezione meccanicistica dello sviluppo del SNC, secondo cui la maturazione
avviene in senso caudo-rostrale per tappe successive, permettendo la motilità volontaria solo dopo il superamento dei riflessi. In
questa concezione si escludono gli aspetti relazionali e motivazionali del movimento. Una lesione delle strutture corticali prima della
loro maturazione la impedisce, e l'intervento terapeutico deve inibire quindi i riflessi e la motilità patologica, e determinarne una
corretta con stimolazioni propriocettive ripetute.

Nei metodi Bobath e Kabat i soggetti con PCI vengono considerati particolari per il loro repertorio di schemi e acquisizioni peculiare.
Il soggetto deve essere collaborante, e l'osservazione deve essere ripetuta costantemente per evidenziare cambiamenti e progressi, anche
nel rapporto con genitori e ambiente. Si considera per la prima volta anche il contesto educativo e il riconoscimento delle figure
significative.

Milani-Comparetti ha portato in evidenza l'ascolto dei genitori, la relazione tra genitori e bambino, e la valorizzazione del bambino
agli occhi dei genitori.

Negli anni 70 si è data importanza alla prevenzione dei rischi, allo studio dei soggetti a rischio e alla diagnosi. Finalmente inizia un
trattamento precoce e un'attenzione particolare alle prime relazioni del bambino con l'esterno. Purtroppo in questo periodo si sono
sviluppati molto interventi diversi, che non si sono integrati subito. 

Negli anni 80 sono aumentate le conoscenze sullo sviluppo del SNC, includendo anche le condizioni ambientali nelle condizioni che
influenzano lo sviluppo, e inserendo la variabilità nelle condizioni normali di sviluppo. Si è definito lo sviluppo del bambino normale
come un'interazione tra schemi innati e esperienze ambientali, che permettono una adattabilità al contesto. Nel bambino con lesione
centrale questa variabilità manca, per mancanza di possibilità di creare schemi alternativi. 

Negli anni 90 si sono sviluppati nuovi studi su movimento e apprendimento motorio, integrando le varie discipline nella ricerca per
integrarle anche nel trattamento.

\paragraph{Il metodo Doman-Delacato} Il metodo Doman-Delacato si basa sull'evocazione delle potenzialità residue nella maggiore quantità
possibile. L'organizzazione è dovuta all'interazione tra genetica e ambiente, con il SNC che regola gli altri sistemi, e integrazione
tra abilità motorie e sensoriali, e un bisogno di partecipazione del bambino nell'ambiente per permettere un apprendimento.
Secondo gli autori i vari problemi dello sviluppo permettono un ritorno in auge dei sistemi inferiori, e il trattamento si attua
attraverso stimolazioni passive corrispondenti al livello di maturazione che la valutazione dà come deficitario. Gli schemi da eseguire
sono in flessione-estensione omolaterale e crociata, con esercizi di strisciamento e gattonamento con bombardamento di stimoli
sensoriali.

\paragraph{Il metodo Peto} Il metodo Peto concepisce la PCI come disturbo dell'apprendimento, che coinvolge anche recezione ed
elaborazione di informazioni e sistemi di controllo. Questo metodo sollecita la sollecitazione spontanea, la partecipazione attiva del
bambino nella vita quotidiana e nelle attività espressive, in relazione alle capacità del singolo.
In questo approccio si sollecitano movimenti finalizzati all'apprendimento, le abilità sottostanti alle performance, e si analizzano i
compiti, rinforzando il comportamento e le sequenze attraverso il linguaggio.

\paragraph{Il metodo Colli-Grisoni} Il metodo Colli-Grisoni si basa sull'apprendimento motorio nel gioco guidato, che deve essere
precoce, coinvolgere la madre nella rieducazione, e si basa sull'iniziativa del bambino e sul suo ruolo attivo per dare un vero
apprendimento.

\paragraph{Il metodo Bobath} Il metodo Bobath è empirico, basato sulle osservazioni del terapista, per poi trarre da queste i principi
teorici. Secondo loro il problema principale è un disturbo della coordinazione dell'attività muscolare, e il movimento volontario si
basa su una piattaforma di integrazione sottocorticale che regola continuamente la postura, secondo il meccanismo posturale riflesso. 
Questo meccanismo ha tre elementi basilari: tono posturale, co-contrazione e pattern di movimento, che se risultano alterati alterano
tutto il meccanismo. Le lesioni cerebrali alterano il meccanismo, impedendo l'esecuzione di movimenti volontari. Per realizzare questo
meccanismo entrano in gioco
\begin{itemize}
\item Reazioni di raddrizzamento: reazioni di adattamento della posizione del corpo per mantenere la posizione del corpo nello spazio e
dei segmenti uno rispetto all'altro, permettendo il raddrizzamento contro gravità e la possibilità di ruotare attorno all'asse corporeo
\item Reazioni di equilibrio: adattamenti automatici per conservare e recuperare l'equilibrio in seguito allo spostamento del centro di
gravità.
\end{itemize}
Queste reazioni sono automatiche e controllate a livello sottocorticale.
I pattern possono essere
\begin{itemize}
\item Primitivi: indicano patologia con tono posturale anormale, se la postura non varia o se gli altri pattern primitivi dello stesso
stadio sono assenti
\item Patologici: 
\begin{itemize}
\item Asimmetria del tronco
\item Tendenza a pronare
\item Arti superiori intraruotati e estesi
\item Arti inferiori intraruotati, estesi e abdotti
\item Flessione plantare delle caviglie
\item Flessione plantare delle dita dei piedi in stazione eretta
\item Estensione della gamba libera con bambino appeso all'altra
\item Estensione e adduzione degli arti inferiori con bambino prono
\end{itemize}
\end{itemize}
Il trattamento si basa su inibizione (dell'attività riflessa abnorme, di posture o pattern primitivi e abnormi stimolando i punti
chiave) e facilitazione dell'attività posturale e motoria corretta, come reazioni di raddrizzamento e equilibrio, con sequenze normali
di organizzazione posturale. Queste stimolazioni si danno tramite la manipolazione.
L'inibizione si attua tramite i pattern inibenti i riflessi, agendo sui punti chiave, e la ripetizione di movimenti normali porta
all'apprendimento e all'uso di questi schemi.
I punti chiave sono parti del corpo ricche di recettori, utili sia per inibire che per facilitare, e possono essere prossimali o
distali, e servono per ridurre il tono e facilitare reazioni posturali e movimenti finalizzati.
Il tapping vuole aumentare il tono con stimolazioni specifiche, stimolando ad esempio gli antagonisti dei muscoli ipertonici, la co
contrazione di agonisti e antagonisti, per reggere i carichi, per migliorare la coordinazione e per stimolare gruppi specifici.

\paragraph{Il metodo Vojta} Il metodo Vojta si basa su \textbf{diagnosi precoce} basata sui fattori di rischio, e va ad agire
preventivamente sulla patologia. Secondo questo concetto, le lesioni alle vie e ai centri efferenti sono dovute a lesioni alle vie e ai
centri afferenti. Questa idea si è sviluppata da degli studi compiuti su scimmie sottoposte ad anossia neonatale, in cui a seconda
dell'età dell'analisi si sono notati danni sempre più estesi a centri cerebrali, e solo a tre anni quadri del tipo spastico, atassico,
atetosico o misto. Secondo questo metodo, quindi, la deprivazione sensoriale colpisce altre attività cerebrali. Ogni movimento è
composto da una componente posturale, di raddrizzamento e di movimento effettivo. Si sono scoperti il rotolamento (evocabile per via
riflessa) riflesso e lo strisciamento (non presente nel neonato normale, ma evocabile per via riflessa) riflesso nel neonato. Secondo
Vojta gli \textbf{stadi di sviluppo} sono inizialmente flessorio, fino a 6 settimane (con riflessi arcaici), poi estensorio (del capo)
fino a 3 mesi, poi flessorio (piedi alla bocca e postura carponi) fino a 8 mesi, e alla fine estensorio, fino a 14 mesi, in stazione
eretta e a quadrupede.
\textbf{La diagnosi precoce} si basa su 7 prove posturali per trovare una futura patologia cerebrale: trazione, sospensione ventrale,
ascellare, collis orizzontal e verticale, prova di peiper e vojta. Se risultano alterate da 1 a 3 prove, la patologia viene indicata
come lieve, fino a 5 come medio lieve, fino a 7 medio grave, se tutte alterate, come il tono, diventa grave.
Il metodo è nato dall'osservazione delle reazioni dei neonati, e dalla presenza di zone grilletto, che se stimolate portavano a risposte
riflesse sempre uguali, che producevano lo strisciamento riflesso.
\textbf{Lo strisciamento riflesso} include gli elementi essenziali della locomozione, attivando meccanismi di sostegno e elevazione, per
appoggiarci e prendere,verticalizzarci e deambulare. Permette di attivare la muscolatura respiratoria, il pavimento pelvico e gli
sfinteri, e di muovere gli occhi.
\textbf{Il rotolamento riflesso}, invece, parte dalla posizione supina, continua in decubito laterale e termina a gatto. Parte di questi
modelli compaiono di solito al sesto mese, altri all'ottavo.
Sfruttando questi due meccanismi si possono inibire gli schemi patologici e facilitare gli schemi normali, stimolando le zone grilletto,
in modo protratto nel tempo e sommando più zone, ripetendo l'esercizio 4 volte al giorno per un quarto d'ora.

\paragraph{Metodo Perfetti-Salvini} Seconodo il metodo Perfetti il movimento è fatto di tre componenti: atto locutivo, con contrazioni
muscolari coordinate, atto illocutivo, per entrare in rapporto col mondo, e atto perlocutivo, per interagire col mondo.
Secondo questo metodo la facilitazione va evitata, perché non aiuta gli ultimi due atti, quindi la finalità e il contesto. Secondo loro
il movimento permette di ottenere notizie dal mondo coi recettori, si recupera apprendendo, è fondamentale l'interazione col mondo, e
devono esserci i problemi conoscitivi. Il contesto è quindi fondamentale, per le informazioni da estrarre, che regolano il movimento
regolando la percezione.
\'E fondamentale far entrare il soggetto in rapporto con qualcosa, per stimolare le varie superifici esploranti, con un problema
percettivo che obblighi all'esplorazione e all'ipotesi percettiva, guidando il soggetto con il linguaggio e controllando attenzione e
memorizzazione.
La valutazione si basa sull'osservazione di alcune azioni, come l'orientamento del capo verso stimoli acustici, il comportamento
oculare, l'esame della coordinazione tra i vari movimenti, l'attività della mano, la coordinazione occhio mano, e l'attività degli arti
inferiori.
Nell'ETC l'esplorazione visiva deve tenere conto di fissazione, inseguimento, movimenti del capo e del tronco, la manipolazione deve
favorire l'interazione con gli oggetti, la frammentazione della superficie esplorante e permettere di controllare sequenze motorie
complesse, e il cammino viene favorito dalla coordinazione occhio-piede, dalla messa in atto di sequenze per acquisire informazioni dal
suolo, e dal trasferimento dinamico del carico. 
