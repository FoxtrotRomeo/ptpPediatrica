\mainmatter
\chapter{Sviluppo neuromotorio del bambino}
Lo sviluppo neuromotorio è un processo complesso, che coinvolge fattori cognitivi, sociali, sensoriali, motori,
emotivi e di relazione. La sua variabilità dipende da fattori genetici, ambientali, dalle esperienze fatte,
dalla costituzione fisica, dal rapporto con le figure di riferimento e dal carattere del bambino.
La genetica influenza lo sviluppo stabilendo dei limiti massimi entro cui questo può avvenire. L'ambiente,
invece, determina quante di queste potenzialità massime vengono effettivamente espresse dall'individuo.

Durante lo sviluppo, i bambini devono sempre essere attivi, curiosi e in cerca di nuove relazioni. Se questo non
dovesse avvenire, il bambino andrà tenuto sotto controllo, valutato e sorvegliato.
Nell'attività di controllo e sorveglianza bisogna prestare attenzione sia al patrimonio genetico, che
all'apprendimento del bambino, dato che il patrimonio genetico influenza l'apprendimento, plasmando le strutture
del bambino, come anche le sollecitazioni che il bambino subisce. Il bambino, d'altro canto, impara muovendosi,
e facendo esperienze nel mondo e con l'adulto che se ne prende cura.
Le competenze motorie del bambino vengono, nel tempo, accumulate sotto forma di schemi motori e posturali nel
sistema nervoso, restando disponibili per richieste successive. Per questo, il movimento può essere inteso anche
come neuromotricità.
D'altro canto, il movimento ha uno scopo, e quindi può essere inteso anche come psicomotricità, a seconda della
funzione che viene svolta. La psicomotricità prende i contenuti della neuromotricità, e vi aggiunge altri
contenuti, che vengono poi automatizzati con l'apprendimento.

Nel tempo, il neonato è stato definito in var modi, che lo etichettavano variamente come \textit{vuoto}, mentre
studi più recenti dimostrano che anche a poche ore dalla nascita è già ricco di competenze, come la discriminazione
di oggetti a 20-30 cm di distanza, la capacità di seguire una pallina dentro il campo visivo, di sentire, di
interagire col mondo, di rispondere agli stimoli e di regolare la coscienza in \textit{sonno profondo, sonno
leggero, dormiveglia, veglia tranquilla, veglia agitata, pianto}. Questi stati comportamentali permettono al bambino
di regolare le proprie interazioni, per rendersi più disponibile all'interrelazione con l'ambiente e gli adulti, e
per eliminare gli input negativi. La relazione con gli adulti e l'ambiente non deve essere vista come una relazione
chiusa di stimolo-risposta, ma come una relazione aperta di proposte e controproposte, che continuano nel tempo.

Pur nel suo essere un bambino competente, il neonato deve essere stimolato e facilitato correttamente per esprimersi
al massimo delle sue possibilità. \\*Lo sviluppo viene descritto in fasi, meno rigide e meno obbligate, sia nei tempi
che nei modi, delle tappe.

\paragraph{Primo trimestre}
Nel primo trimestre, il bambino si adatta all'ambiente extrauterino, riorganizza le funzioni neuromotorie e gli stati
comportamentali, organizza le competenze alimentari, e matura le competenze di relazione e il linguaggio.

Per la parte motoria deve superare i riflessi arcaici, l'asimmetria iniziale, deve essere in grado di estendere il
capo in posizione prona, controllare il capo, deve mettere le mani in bocca, coordinare capo e occhi, e sgambettare
alternatamente sia in posizione supina che prona. 
Per la parte cognitiva deve essere in grado di seguire per 180 gradi un oggetto, percepire le variazioni nel movimento
del capo, esplorare lo spazio con le mani, e esplorare con la bocca e le mani il proprio corpo.
Per la parte affettiva deve essere in grado di sviluppare il sorriso come risposta a stimoli piacevoli, di coordinare
il pianto come risposta a stimoli spiacevoli, di essere più o meno tranquillo in base alla stabilità posturale e alle
condizioni ambientali, e può sviluppare reazioni nevrotiche.

In questa fase, la capacità di allineare e disallineare capo e tronco è la principale funzione motoria, e permette
l'esplorazione visiva dell'ambiente, che è la principale funzione cognitiva. La diagnosi si concentra sulla capacità di
mantenere la stabilità e la flessione, e sulle capacità di relazione col mondo esterno.

\subparagraph{Patologia}
Deficit organici o carenze nell'ambiente portano a alterazioni senso-motorie, a ritardo nel controllo del capo, e a
distorsioni nella conoscenza dello spazio, degli oggetti e delle relazioni.
Si riduce l'esplorazione del corpo e la ricerca del capezzolo, i movimenti sono rigidi e non si interagisce con lo sguardo
dell'adulto. Questo porta a alterazione delle informazioni in entrata, e a problemi relazionali.
Un trattamento precoce porta a una migliore maturazione delle abilità di movimento, cognizione, comunicazione, e della
sicurezza.

\paragraph{Secondo trimestre}
Nel secondo trimestre inizia la separazione dall'adulto, aumenta l'interesse per l'ambiente esterno, e inizia lo
spostamento autonomo.
Per la parte motoria compare il rotolamento, matura la presa volontaria, emerge la coordinazione occhio-mano, si osservano
e si succhiano parti del corpo, e si controllano meglio la respirazione e l'emissione di suoni.
Per la parte cognitiva, maturano le reazioni circolari, si identificano proprietà dell'oggetto, si fa più attenzione a
vari elementi del proprio corpo, e si elabora il movimento come mezzo per superare le distanze.
Per la parte affettiva, si riconosce un viso noto, inizia l'espressività mimica, si cercano oggetti piacevoli, e diventa
possibile il riso in risposta al gioco.

In questa fase, la principale funzione motoria è il controllo del tronco da prono, la presa e il controllo visivo. Si
diventa stabili in prono e in supino, e si inizia un controllo da seduti.

\subparagraph{Patologia}
La persistenza dei riflessi arcaici riduce la possibilità di esplorare corpo, oggetto e ambiente, lasciando distorsioni
cognitive.
La stimolazione ambientale ridotta e il mantenimento della posizione supina portano a estensione e abduzione della braccia,
con movimento scarso. Questo porta a ritardi nell'acquisizione della motricità fine, della prassia e dell'organizzazione
grafo-motoria.

\paragraph{Terzo trimestre}
In questa fase migliorano: il controllo da seduto, i passaggi posturali, la manipolazione, il linguaggio, e sviluppa
l'angoscia dell'estraneo.

Per la parte motoria si sviluppano la capacità di strisciare, di gattonare e di stare seduto, maturano le reazioni di
equilibrio, si riesce a mettersi seduto in autonomia e a stabilizzarsi, migliora la manipolazione e compare la lallazione.
Per la parte cognitiva, si sviluppa la distinzione tra io e oggetto, si ampliano le esperienze, compare la reazione
circolare terziaria, si gioca con oggetti e adulti, e si inizia a sapere mangiare da solo.
Per la parte affettiva, si riconoscono le persone note e non, il pianto e il riso sono motivati come reazioni affettive, e
ci sono risposte verbali affettive alle sollecitazioni ambientali.

Emergono la stazione eretta, con appoggio, la stazione seduta senza appoggio, e lo spostamento a gattoni. La manipolazione
si sviluppa da seduto, grazie alla mancanza di appoggi obbligatori.

\subparagraph{Patologia}
Gli atteggiamenti patologici impediscono conoscenza, esplorazione e apprendimento. Possono essere mancanze dovute anche a
ambienti carenti o atteggiamenti sbagliati, come il box, che limita l'acquisizione della sicurezza. \'E fondamentale anche
l'esplorazione continua dell'oggetto e dell'ambiente.
Importante imporre dei limiti, per l'educazione del bambino, con il no.

\paragraph{Quarto trimestre}
In questa fase inizia il processo di individuazione e separazione, l'indipendenza motoria, la maturazione della presa e la
comunicazione attraverso il linguaggio.

Per la parte motoria si acquisiscono il cammino e la posizione statica autonoma, e matura la presa, con rilascio volontario
e adattamento della mano all'oggetto.
Per la parte cognitiva, si usa la manualità in modo finalizzato, si usano gli oggetti per la loro funzione, si cercano
relazioni causa-effetto, si diventa più autonomi nella cura di sè e si comprendono comandi specifici.
Per la parte affettiva si inizia a comunicare attraverso il linguaggio, si cerca la presenza dell'altro attraverso il gioco,
e si richiama l'attenzione dell'adulto.

Migliora lo spostamento a gattoni, migliorano i cambi di posizione da seduto ad altre posizioni, e progressivamente, fino a
18 mesi, si abbassa la guardia, fino a diventare capace di usare le mani mentre cammina per fare altro o oscillarle.

\subparagraph{Patologia}
Le difficoltà motorie impattano molto su presa e rilascio, e sul linguaggio. L'intervento è ormai tardivo. Il movimento è
essenziale per apprendere e interiorizzare, e se manca fino ad ora è davvero grave.

\paragraph{Sviluppo della presa}
Alla nascita i riflessi impediscono la motilità volontaria. Con l'attenuarsi del grasp si presenta la presa volontaria, di
tipo cubito-palmare e con i vari movimenti dell'arto superiore.

A quattro mesi ancora non è presente una dominanza, e con le mani esplora il suo corpo e l'ambiente esterno, con la
coordinazione occhio-mano.

A cinque mesi non è del tutto scomparso il fenomeno del grattage, con cui il bambino gratta il piano con le dita prima di
afferrare l'oggetto.

A sei-sette mesi la presa diventa radio-palmare, con il pollice usato sullo stesso piano delle altre dita, e il polso
dritto, e una iniziale lateralizzazione. Tramite le successive prove ed errori, il bambino adatta sempre meglio la presa.
Il bambino ha curiosità delle sue mani.

Ad otto mesi si usa il pollice per un movimento di pinza inferiore. La mano si apre ancora troppo rispetto all'oggetto da
afferrare.

Tra nove e undici mesi si usa l'indice assieme al pollice, per raccogliere oggetti piccoli. Il gioco è sempre più organizzato
e si sviluppano i primi gesti.

A un anno la mano è indipendente dal cingolo scapolo-omerale, il pollice e l'indice lavorano in opposizione come dita più
importanti, si estende il polso, e si usano le dita una alla volta. La presa è di tipo "pinza superiore".
