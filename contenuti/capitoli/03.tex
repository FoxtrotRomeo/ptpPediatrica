\chapter{Riabilitazione in età evolutiva}
La prassi riabilitativa va aggiornata sulla base delle nuove conoscenze rese disponibili dalla scienza. In particolare,
chi fa clinica deve passare da un approccio che dalla clinica cerca una spiegazione, a un approccio che dalla teoria
faccia arrivare ad una prassi corretta.
Dalla scienza ci si aspetta un modello che permetta di spiegare i disordini che conseguono a una lesione, per poter
avere quindi una prassi razionale e corretta.

Il controllo motorio è l'insieme dei processi che permettono di mantenere la stabilità e l'equilibrio in statica e durante
movimenti più o meno complessi. Include sistemi percettivi, sistemi che regolano azioni più o meno complesse, e sistemi
cognitivi, per definire gli aspetti di intenzione, attezione e affettivi.
Le azioni organizzano le nostre percezioni, che rendono più corretti e precisi i movimenti. Le percezioni, comunque, dirigono
le nostre azioni, permettendone una corretta organizzazione e svolgimento.
La percezione è sia raccolta delle informazioni, che trasmissione ed elaborazione. Permette di organizzare internamente il
movimento, e tutti i sensi devono essere integrati.
\'E importante anche l'integrazione tra individuo, ambiente e compito.

Secondo la teoria \textbf{neuromaturazionista} l'input sensoriale determina la risposta motoria, e i riflessi sono alla base
del comportamento motorio dell'individuo. Le funzioni vengono sviluppate per stadi, e esiste una organizzazione gerarchica,
con i centri corticali che controllano i sottocorticali, che controllano gli spinali.
Alcuni metodi, come il Doman, il Vojta e il Bobath, si basano sull'inibizione dei riflessi patologici e la facilitazione di
risposte riflesse corrette, ma in questo modo non si ha un miglioramento della funzione, ma solo delle risposte riflesse.

Secondo la teoria \textbf{sistemica} è presente un'interazione tra vari sistemi, come quello nervoso, muscolare, scheletrico,
e con l'ambiente. Gli input sensoriali, poi, hanno il compito di regolare e adattare cambiamenti e azioni. Le informazioni
vengono poi selezionate in base a quali siano più adatte per svolgere la funzione.

L'apprendimento motorio è una modificazione del comportamento motorio, con acquisizione di nuove abilità e un processo che
coinvolge percezione, movimento e cognizione, per risolvere problemi tra individuo e ambiente. L'apprendimento è programmato
geneticamente, basato sull'esposizione a esperienze varie e significative. Grazie a processi associati con l'esperienza si
inducono cambiamenti permanenti nelle capacità di svolgere compiti motori.

Secondo la teoria \textbf{neuromaturazionista} l'apprendimento è dovuto alla maturazione di strutture nervose dal midollo alla
corteccia, e la ripetizione passiva di movimenti normali porta al loro trasferimento automatico in apprendimento e attività.

Secondo la teoria \textbf{sistemica}, l'apprendimento motorio è frutto di una interazione tra capacità già presenti e ambiente,
che fornisce l'occasione e il modo per risolvere i problemi. Lo sviluppo dipende, quindi, dall'organizzazione di pattern, che
diventano sempre più adattivi con la ripetizione. Solo con l'acquisizione delle abilità di base si potranno poi acquisire abilità
più avanzate. Il controllo motorio, in questo caso, è indispensabile per un apprendimento stabile e svincolato dal contesto.

L'approccio cognitivista dice che è importante interagire con l'ambiente per sviluppare nuove capacità motorie. Interagendo con
l'ambiente emergono desideri ed emozioni, che portano all'intenzione di compiere un'azione, stabilendo un piano d'azione in base
alle nostre percezioni interne ed esterne.
Il piano d'azione viene generato di volta in volta per le circostanze particolari in cui si svolge l'azione, e viene attuato con
delle strategie innate, ma può essere corretto o modificato grazie a nuove afferenze. In quest'ottica, ripetere i movimenti in
circostanze coerenti con l'ambiente di vita del bambino facilita e migliora l'apprendimento e l'acquisizione di nuove strategie.

Per poter mettere in atto queste strategie, il bambino dovrebbe essere in possesso di strategie di movimento innate (riflessi,
reazioni automatiche e pattern di movimento), essere in grado di analizzare e selezionare le informazioni (percezione e
contestualizzazione degli stimoli), avere delle regole per elaborare le strategie cognitive e adattive per le richieste sempre
diverse, e avere un sistema di controllo (feed-forward, feed-back performance e feed-back risultato) per controllare
l'appropriatezza dei piani d'azione.
In base a questi requisiti, i problemi motori possono avere tre cause: cognitivo o depressivo, con mancanza di intenzione,
dispercettivo, con informazioni errate, o motorio.

Secondo le teorie di Gordon, il processo terapeutico deve includere:
\begin{itemize}
\item Processi di controllo motorio, in cui si aiuta il bambino a risolvere problemi motori e a sviluppare abilità predittive e
anticipatorie, vedendo il bambino come attivo nella soluzione dei problemi e guidandolo nell'analisi delle informazioni e nella
pianificazione delle strategie.
\item Sviluppo delle abilità motorie, in cui si guida il bambino a risolvere un problema motorio in contesti vari, il più
possibile vicini alla vita del bambino, e vari per permettere una generalizzazione. Le proposte terapeutiche devono essere
elaborate in modo da includere tutti gli elementi che vengono attivati nell'apprendimento, e sarebbe opportuno che le proposte
venissero elaborate anche nel vero ambiente di vita del bambino.
\item Patogenesi del disordine motorio, che dipende da uno o più sistemi danneggiati. Con un intervento precoce, si evitano
compensi inadeguati e si sfruttano al massimo le potenzialità del bambino.
\item Riorganizzazione e recupero funzionale: il trattamento a cui sottoponiamo il bambino non ripercorre lo sviluppo normale,
ma piuttosto favorisce l'esplorazione dell'ambiente nel rispetto delle caratteristiche del singolo soggetto, per promuovere
l'apprendimento di soluzioni.
\end{itemize}

Al momento non c'è una singola metodica riconosciuta valida, per la varietà dei quadri clinici a cui ci si trova di fronte e la
complessità dei problemi.
Nella \textbf{riabilitazione ecologica} si evidenzia l'importanza della percezione per apprendere nuove abilità motorie, creando
compiti di interazione con l'ambiente per aumentare la flessibilità del bambino. Per questo, l'ambiente va modificato, cercando
l'interazione tra il bambino e l'ambiente e la società.

Nell'\textbf{apprendimento motorio nel gioco guidato} ci si basa su una definizione dei livelli coinvolti nel disordine, per definire
obiettivi e modalità del progetto riabilitativo, con lo scopo di apprendere e controllare il movimento, mettendo a disposizione
dei tool diagnostici. L'intervento terapeutico terrà in considerazione:
\begin{itemize}
\item Il livello delle intenzioni, degli scopi e della formulazione del piano d'azione, per promuovere l'interazione, e sostenere
e guidare il bambino nella raccolta di informazioni, nella formulazione del piano e delle ipotesi per metterlo in atto.
\item Il livello di strategie e mezzi per risolvere il problema, per guidare il bambino nella scelta dei pattern più idonei, e
nell'analisi dei risultati e degli errori
\item Il livello degli strumenti neuromotori, per prevenire le retrazioni tendinee, risurre spasticità e ipercinesie, e correggere
i deficit sensoriali.
\end{itemize}
In questo modello, diagnosi e intervento vanno calibrati in base alle caratteristiche individuali del bambino.
La strategia terapeutica deve essere multisettoriale, \textit{proponendo} le attività giuste per evocare iniziative adeguate,
\textit{sostenendo} l'attenzione del bambino mentre formula il piano d'azione, \textit{aspettando} i suoi tempi e modi nella scelta
di strategie e schemi, \textit{favorendo} l'attuazione del programma motorio, \textit{guidandolo} nell'analisi dei risultati e
\textit{variando} l'ambiente per permettere una variazione delle strategie.
Viene utilizzato in bambini fino ai sei anni, prima con la madre, poi in gruppi con patologie simili, ma sempre con un terapista
per ogni bambino.
Il programma terapeutico deve includere una \textit{valutazione} iniziale, \textit{l'analisi} dei dati raccolti, la
\textit{definizione} degli obiettivi a breve e a lungo termine, lo \textit{sviluppo} del piano di trattamento, la sua
\textit{attuazione} e una \textit{rivalutazione} degli outcome.
Per poterlo fare, è necessario conoscere lo sviluppo normale delle funzioni, i disordini a cui vanno incontro nelle forme cliniche,
i disordini del singolo bambino, definire un progetto terapeutico (per far conoscere il disordine e risolvere i problemi),
considerare tutti gli strumenti possibili, modulare il progetto sul singolo bambino e verificare l'idoneità del progetto in base
ai risultati.

Entrambi questi approcci hanno portato a modificazioni della prassi terapeutica, permettendo ai bambini di sperimentare strategie
che il bambino capisce essere le risposte ottimali al problema, come è obiettivo del trattamento. \'E fondamentale cercare di
ottenere un carry over delle abilità apprese nella vita di tutti i giorni. Il terapista deve diventare un problem solver attivo,
per trovare, in base alle sue conoscenze, modalità e strategie per raggiungere obiettivi funzionali nel paziente.

Il \textbf{mental training} ci insegna come l'immagine abbia attivazioni cerebrali simili all'azione, permettendo il recupero già
in una prima fase dopo la lesione cerebrale. Si può usare nei bambini collaboranti, aiutandoli a raccogliere le informazioni
fondamentali, elaborarle e elaborare strategie per risolvere i problemi.

I \textbf{neuroni specchio} sono un sistema che permette di imparare per imitazione, scomponendo le attività complesse osservate
in atti semplici, e permettendo di ricomporre questi atti in nuove sequenze.
