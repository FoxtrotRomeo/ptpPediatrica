\chapter{Ritardo motorio e psicomotorio}
\textbf{Il ritardo motorio} è un rallentamento delle sequenze di sviluppo delle competenze motorie globali, dell'organizzazione
antigravitaria della postura e degli schemi di spostamento che vengono di solito acquisiti nei primi due anni di vita.
Nella diagnosi differenziale bisogna escludere altre patologie che possano portare a un ritardo delle acquisizioni motorie, come
problemi sensoriali, lesioni nervose e muscolari, dismorfie e problemi comunicativi e cognitivi. \'E caratterizzato dal ritardo della
sequenza delle competenze motorie globali, che non coinvolge funzioni più fini.
Viene considerato l'espressione di un disordine evolutivo, che coinvolge più sistemi, come caratteristiche del sistema
muscoloscheletrico che rendono più difficile il controllo della postura, o poche possibilità di sperimentare e di esercitarsi nel
controllo. Può essere secondario a malattie o ricoveri prolungati, che portano ad alterazioni reversibili con la reversione della
patologia che le provoca.
\textbf{La terapia} passa attraverso l'educazione posturale, il coinvolgimento della famiglia, l'aumento dell'esplorazione, e la
sperimentazione di strategie per risolvere i problemi.
\textbf{Il ritardo psicomotorio} è la mancata acquisizione delle competenze motorie, cognitive e comunicative in relazione all'età
cronologica. Comporta un ritardo armonico delle funzioni adattive, che non è sempre definibile prima dei tre anni.
\'E una sindrome aspecifica con patogenesi varia. Può precedere le manifestazioni del ritardo mentale, di cui non si parla prima dei tre
anni.
Si presenta inizialmente con una ipotonia globale, ritardo nell'acquisizione delle varie tappe, nella comparsa delle interazioni sociali
e della mimica, scarsa iniziativa in esplorazione e sperimentazione e comunicazione povera. Dal secondo anno in poi manca la comparsa
delle prassie, del problem solving e dei giochi imitativi. I bambini imparano per imitazione, senza precisione e con stereotipie.
A 15 mesi emergono un'esitazione per la deambulazione, la riattivazione di modalità primitive di movimento, e la povertà
dell'iniziativa.
A 3 anni dimostrano di essere impacciati nei compiti complessi, e una difficoltà a usare il movimento per apprendere.
A 4 anni non riescono a mescolare compiti semplici e complessi, e hanno una manipolazione primitiva.

Si presentano dei deficit in almeno due di queste aree di sviluppo: motricità fine, grossolana, linguaggio, capacità sociali, ADL. Il
ritardo nelle competenze motorie grossolane evolve favorevolmente, con un ritardo delle posture, mentre resta un disordine nelle abilità
motorie fini e complesse, che mostra un disordine cognitivo, che interferisce nel controllo motorio.

La terapia varia a seconda di quali siano le patologie alla base del ritardo, di solito con un intervento riabilitativo precoce per
migliorare la comunicazione con la madre e la sperimentazione del bambino, e per migliorare la selezione delle informazioni rievanti nei
vari compiti.
Un QI minore di 70 indica un funzionamento molto sotto la media. In particolare, il ritardo è
\begin{itemize}
\item Lieve tra 55 e 70: include la maggior parte dei soggetti, senza gravi anomalie del linguaggio e con difficoltà che si presentano a
scuola. Possono arrivare a un livello da quinta elementare attorno ai 20 anni.
\item Moderato tra 40 e 50: La maggior parte acquisisce capacità comunicative nella prima fanciullezza, di solito non arrivano oltre un
livello da seconda elementare, e hanno difficoltà con le convenzioni sociali.
\item Grave tra 20 e 40: Acquisiscono un livello di linguaggio minimo, o nessun linguaggio nella prima fanciullezza, ma possono imparare
a parlare ed essere addestrati alla cura della persona durante la scuola. Hanno sempre bisogno di un ambiente che li protegga e li
stimoli.
\item Gravissimo sotto i 20: Molto rari, hanno un funzionamento sensomotorio compromesso, poca autonomia nelle ADL, linguaggio molto
limitato o inesistente, e sono sempre dipendenti dagli adulti.
\end{itemize}

Il ritardo è spesso l'espressione di sindromi genetiche, neuromuscolari, PCI, o encefalopatie evolutive metaboliche o degenerative.
