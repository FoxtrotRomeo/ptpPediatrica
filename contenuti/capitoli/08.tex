\chapter{La presa in carico}
Il progetto terapeutico diretto al bambino si realizza tramite la collaborazione di varie figure, professionali e non, tra loro e con il
bambino.
Il percorso inizia con la presa in carico, un momento delicato, per le condizioni psicologiche dei genitori che hanno appena appreso la
diagnosi.
L'equipe deve, quindi, garantire accoglienza, disponibilità, rispetto dei tempi mentali dei genitori, e deve presentare i sintomi della
patologia poco alla volta.

\'E importante osservare il bambino assieme ai genitori, per rilevare gli aspetti positivi, capire i messaggi del bambino e
interpretarne i bisogni, e far capire ai genitori che il bambino che avevano immaginato è molto diverso da quello reale.
Spesso le madri hanno difficoltà a relazionarsi con i bambini, perchè la loro immagine del bambino è molto diversa da quella reale. Per
questo, le madri tentano in tutti i modi di riparare la funzione mancante, dimenticandosi le altre necessità del bambino, o avendo
difficoltà a comprenderle. Questo provoca un impoverimento del rapporto comunicativo tra madre e figlio.

Il terapista deve guidare questa relazione, in modo da permettere al bambino di ricevere le informazioni estero e propriocettive più
adeguate e di partecipare alla vita di tutti i giorni. Il fisioterapista deve anche garantire il contatto corporeo tra madre e bambino,
e far recepire alla madre i messaggi del figlio. Deve, inoltre, permettere alla madre di vedere le potenzialità del figlio.

La madre spesso non riesce a separarsi dal bambino, per via del deficit motorio, cognitivo, della scarsa crescita e adattabilità del
bambino a situazioni nuove, e del senso di iperprotezione che prova.
Il processo di separazione, normalmente, passa attraverso delle fasi:
\begin{itemize}
\item Differenziazione: questa fase va da 6 a 10 mesi, e il bambino inizia a percepire il proprio corpo come separato da quello della
madre, senza restare in braccio passivamente, ma cercando di esplorare l'ambiente.
\item Sperimentazione: questa fase va da 10 a 16 mesi, con il bambino che esplora autonomamente l'ambiente, con progressiva
consapevolezza del suo corpo, capendo il concetto della distanza e del distacco dalla madre. La presenza della madre consente una sorta
di ricarica.
\item Riavvicinamento: questa fase va da 16 a 24 mesi, con il bambino che vuole condividere le sue esperienze con la madre, e ha
angoscia per la separazione.
\item Individuazione: questa fase avviene nel terzo anno, con il bambino che afferma la propria identità e si separa dalla madre.
\end{itemize}

Per favorire il processo possiamo inserire ausili per la posizione seduta, in modo da creare sostegno e contenimento come nelle braccia
della madre, e da dare sicurezza al bambino e alla madre. \'E importante anche favorire lo spostamento, per permettere al bambino di
esplorare l'ambiente, di allontanarsi dalla madre e di tornare da lei, con un ausilio specifico. Questo ausilio deve permettere di
muoversi a una buona velocità e senza troppa fatica, per evitare difficoltà e frustrazione che porterebbero a interrompere i tentativi
prima del tempo. 

I genitori devono capire che l'ausilio aiuta il bambino a fare le esperienze più normali possibile, e aiuta loro a prendersene cura. Il
momento in cui si propone l'ausilio è fondamentale, l'inserimento deve essere graduale, e l'ausilio va provato nelle sedute di terapia.
L'ausilio deve essere gradevole esteticamente, e deve coinvolgere i genitori nella scelta.

Ascoltare i bambini e i genitori è il miglior modo per fare un lavoro migliore la prossima volta.

\chapter{Gli ausili per la vita quotidiana}
Le attività dell'uomo sono la parte visibile di quello che succede nel suo mondo interiore. Il gesto viene gestito dal sistema
sensomotorio, adattandolo al mondo attorno a noi, ma in caso di danno non ci si adatta più così bene, dovendo creare nuove strategie per
raggiungere l'obiettivo.

Esistono molti modi per fare la stessa azione, e visto che l'importante è compiere l'azione, si può adattare il gesto agli oggetti, o si
possono adattare gli oggetti su cui lo si compie al gesto. Anche i luoghi, fisici e sociali, influenzano le azioni che vengono compiute.
Le ADL si possono dividere in:
\begin{itemize}
\item Cura di sè, come
	\begin{itemize}
	\item Igiene personale: fondamentale per mantenersi in salute e scandire la giornata.
	\item Abbigliarsi: per l'abbigliamento si usano tessuti con scarsi attrito, taglie abbondanti, chiusure grandi che facilitano la
  presa, bottoni grandi o velcro, e scarpe comode con chisura a strappo. Si possono fornire degli ausili anche per facilitare la presa
  dei vari pezzi di vestito.
	\item Alimentarsi: esistono ausili per facilitare la presa e le varie attività, in modo da non dover compiere i movimenti più
  difficili, o da facilitarli. Gli ausili possono essere specifici, o appartenere al mercato aspecifico.
	\item Attività domestiche varie: la casa può venire adattata, in modo da essere più accessibile nelle sue parti, e da rendere più
  facili le varie attività che vi si svolgono.
	\end{itemize}
\item Attività professionali e di relazione: sono attività che permettono di avere contatti con altre persone attraverso la
comunicazione. Possono essere facilitate da ausili e tecnologie informatiche, anche nelle disabilità gravi. Le attività professionali
permettono sia di svolgere la propria professione, sia di avviare un percorso formativo per garantire la realizzazione di sè.
\item Tempo libero: sono attività anche molto varie, che vengono aiutate da ausili altrettanto vari.
\end{itemize}

Gli ausili permettono una maggiore autonomia, quindi una maggiore capacità di governare sè stessi, con un equilibrio tra le relazioni
con sè stessi, con l'ambiente e con gli altri, permettendoci di svolgere scelte autonome, muovendoci nell'ambiente e facendo le nostre
attività, secondo le nostre necessità, e interagendo con gli altri con modalità e intensità di nostro gradimento.
Questo ci permette di esprimere noi stessi, sviluppando e mettendo in pratica i nostri valori, con un maggiore inserimento sociale.

Perchè questo possa avvenire, servono
\begin{itemize}
\item Ambiente accessibile: adattando l'ambiente alle persone che ne fruiscono
\item Ausili tecnici: sono strumenti che permettono di adattare la persona all'ambiente
\item Assistenza personale: è l'aiuto fisico che altri forniscono per aiutare lo svolgimento di alcune attività quotidiane
\end{itemize}

\'E importante far capire al paziente che gli ausili e gli adattamenti dell'ambiente non impediscono le interazioni con le altre
persone, ma le cambiano, aggiungendo tempo, spazio e modi per renderle più piacevoli e meno vincolate alla patologia.
Gli ausili prevengono, compensano e alleviano le menomazioni, le limitazioni alle attività o gli ostacoli alla partecipazione. Possono
essere prodotti appositamente per questo scopo, o essere disponibili in commercio normalmente.

Le soluzioni assistive includono soluzioni per la postura, lo spostamento, le ADL, la comunicazione, gli adattamenti ambientali, la
domotica e le tecnologie multimediali. Esiste una classificazione internazionale degli ausili, creata nel 2007, che li divide a seconda
della loro funzione, con un obiettivo di autonomia ben specifico.

Gli ausili hanno tre dimensioni: l'attività per cui si cerca l'autonomia, l'ambiente in cui va svolta, e l'utente, con tutte le sue
caratteristiche.

\chapter {Le malattie neuromuscolari}
Hanno decorso progressivo, e possono avere un danno primario, se danneggiano direttamente il muscolo, o secondario, se danneggiano
invece il SNP, e, attraverso questo, portano a degradazione del muscolo. A seconda di questo, si dicono miopatie o neuropatie.

\paragraph{Le miopatie}
Le miopatie possono essere legate al cromosoma X, possono essere autosomiche recessive, dominanti, metaboliche, congenite o miotoniche.

La distrofia di Duchenne è correlata al cromosoma X, si manifesta attorno ai 3 anni, diventa molto evidente a 5, e porta a contratture,
deambulazione sulle punte, aumento della lordosi e perdita di forza progressiva a 6. A 8 anni si rende necessario deambulare con ausili,
a 12 si perde il cammino, con le retrazioni che peggiorano, la scoliosi che avanza e la respirazione che viene impedita dal progredire
della scoliosi, portando a volte a infezioni polmonari attorno ai 18 anni. Nella riabilitazione, si cerca di mantenere l'autonomia per
quanto possibile, assieme alla funzionalità muscolare, prevenendo le retrazioni e le deformità e l'atrofia delle fibre muscolari. Si
cerca anche di potenziare la funzione respiratoria, e di evitare il sovrappeso.

La terapia chirurgica è indicata, come tenotomia dei flessori delle ginocchia e del gastrocnemio, per prolungare la deambulazione,
stabilizzando la scoliosi al superamento dei 25-30 gradi di Cobb.

La riabilitazione incoraggia l'attività muscolare, riduce il deterioramento muscolare, e mira ad aumentare le attività, con adattamenti
per avere una vita attiva. Vanno evitati gli esercizi di resistenza, che accellererebbero il deterioramento muscolare.

Le ortesi notturne permettono di ridurre la limitazione della dorsiflessione della caviglia, e i tutori da carico permettono di
ritardare le contratture e prolungare la deambulazione, nei soggetti deambulanti, mentre in quelli non deambulanti è importante
utilizzare ausili per la posizione seduta.

Esistono anche ausili per la statica, soprattutto subito dopo la perdita della deambulazione, e si consiglia la carrozzina elettronica
per l'autonomia degli spostamenti, carrozzina che deve permettere basculamento, inclinazione e poggiatesta con posizioni regolabili.

\paragraph{Le neuropatie}
Le neuropatie sono principalmente atrofie muscolari spinali, di tipi 1, 2, 3, o 4, e le neuropatie sensitivo-motorie specifiche a
trasmissione variabile.

L'atrofia muscolare spinale è una malattia degenerativa genetica autosomica recessiva, che porta a perdere i secondi motoneuroni,
denervando poco a poco i muscoli e portando all'atrofia muscolare senza fibrosi, con perdita della forza.
Si divide in 4 forme:
\begin{itemize}
\item Tipo 1: insorge precocemente, impedisce la stazione seduta autonoma e la respirazione. A seconda dell'epoca di esordio si può
suddividere ancora.
\item Tipo 2: insorge prima dei 18 mesi, rende impossibile la stazione eretta in autonomia, ma permette per un po' la stazione seduta in
autonomia. Ulteriormente suddivisa in 10 forme.
\item Tipo 3: Insorge dopo i 18 mesi, ma prima dell'età adulta. permette, almeno per un po', la stazione eretta e la deambulzione.
\item Tipo 4: insorge in età adulta, solo con lievi limitazioni motorie.
\end{itemize}

Nelle prime tre forme, le funzioni cognitive restano integre, si ha una paresi flaccida dei quattro arti, e una progressiva perdita
della respirazione. La disfagia peggiora col passare del tempo, e si riducono anche le capacità comunicative. La continenza sfinterica
viene conservata, mentre nelle fasi avanzate e nelle forme gravi ci sono stipsi e reflusso gastroesofageo. Possono presentarsi dolore e
scoliosi per le retrazioni dei muscoli, e il soggetto è sempre più dipendente nelle ADL.

Nelle prime due forme è indicata la fisioterapia respiratoria, ed eventualmente un aiuto ventilatorio per le ore notturne, e si deve
supportare l'evacuazione. Sono fondamentali anche la mobilizzazione quotidiana per le retrazioni e per evitare deformità ossee. 

Nei casi più gravi del tipo 1, si può procedere anche con intubazione e ventilazione invasiva, ed eventualmente PEG per supportare le
funzioni più compromesse. 
